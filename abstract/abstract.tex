\documentclass[11pt, letterpaper]{article}
\usepackage[utf8]{inputenc}
\usepackage{caption} % for table captions
\usepackage{amsmath} % for multi - line equations and piecewises
\usepackage{graphicx}
%\usepackage{textcomp}
\usepackage{xspace}
\usepackage{verbatim} % for block comments
%\usepackage{subfig} % for subfigures
\usepackage{enumitem} % for a) b) c) lists
\newcommand{\Cyclus}{\textsc{Cyclus}\xspace} %
\newcommand{\Cycamore}{\textsc{Cycamore}\xspace} %
\usepackage{tabularx}
\usepackage{color}
\usepackage{setspace}
\definecolor{bg}{rgb}{0.95, 0.95, 0.95}
\newcolumntype{b}{X}
\newcolumntype{f}{ > {\hsize=.15\hsize}X}
\newcolumntype{s}{ > {\hsize=.5\hsize}X}
\newcolumntype{m}{ > {\hsize=.75\hsize}X}
\newcolumntype{r}{ > {\hsize=1.1\hsize}X}
\usepackage{titling}
\usepackage[hang, flushmargin]{footmisc}
\renewcommand *\footnoterule{}
\graphicspath{{images /}}

\title{Simulation of spent nuclear fuel loading into a final waste repository
        \\ \vspace{0.5em} WM Symposia Poster Abstract}
\author{Gwendolyn J. Chee}


\begin{document}
	\maketitle
	\hrule

\section * {}
\doublespacing
The largest barriers facing nuclear waste management in the U.S. 
are the political and social obstacles towards siting a final 
waste repository. 
It has been shown that permanent underground disposal of nuclear 
waste provides excellent isolation from the human-inhabited 
environment \cite{rechard_evolution_2014}. 
Therefore, this work relies on the expectation that the chosen 
method of long term disposal of spent nuclear fuel (SNF) will be 
a deep geologic repository and that a site will eventually be selected.
In this work, U.S. historical SNF inventory data 
\cite{peterson_unf-st&dards_2017} is used in various simulations that model 
different transfer and loading strategies for moving SNF from reactor 
sites to a final waste repository.  
First-in-first-out and last-in-first-out fuel allocation strategies 
are considered. 
The goal of this work is to determine which strategy 
best maximizes mass loading of the waste repository. 
These simulations are performed using Cyclus, an \textit{agent-based} 
fuel cycle simulation framework. In Cyclus, each facility in 
the fuel cycle is modeled individually and they interact with one another 
as independent \textit{agents}. 
For this work, a waste conditioning facility agent 
and a simple heat-limited repository agent were created. 
The conditioning facility packages spent fuel assemblies into a waste 
canister that has user defined properties such as radius length and material 
thermal conductivity. 
The repository facility emplaces waste canisters by 
maximizing mass loading while remaining below the thermal 
limit of the host geologic media. 




\bibliographystyle{unsrt}
\bibliography{bibliography.bib}

\end{document}
