%%%%%%%%%%%%%%%%%%%%%%%%%%%%%%%%%%%%%%%%%
% Jacobs Landscape Poster
% LaTeX Template
% Version 1.1 (14/06/14)
%
% Created by:
% Computational Physics and Biophysics Group, Jacobs University
% https://teamwork.jacobs-university.de:8443/confluence/display/CoPandBiG/LaTeX+Poster
% 
% Further modified by:
% Nathaniel Johnston (nathaniel@njohnston.ca)
%
% This template has been downloaded from:
% http://www.LaTeXTemplates.com
%
% License:
% CC BY-NC-SA 3.0 (http://creativecommons.org/licenses/by-nc-sa/3.0/)
%
%%%%%%%%%%%%%%%%%%%%%%%%%%%%%%%%%%%%%%%%%

%----------------------------------------------------------------------------------------
%	PACKAGES AND OTHER DOCUMENT CONFIGURATIONS
%----------------------------------------------------------------------------------------

\documentclass[final]{beamer}

\usepackage[scale=1.0]{beamerposter} % Use the beamerposter package for laying out the poster
\usepackage[acronym,toc]{glossaries}
\include{acros}
\usetheme{confposter} % Use the confposter theme supplied with this template

\setbeamercolor{block title}{fg=dblue!80,bg=white} % Colors of the block titles
\setbeamercolor{block body}{fg=black,bg=white} % Colors of the body of blocks
\setbeamercolor{block alerted title}{fg=white,bg=dblue!70} % Colors of the highlighted block titles
\setbeamercolor{block alerted body}{fg=black,bg=dblue!10} % Colors of the body of highlighted blocks
% Many more colors are available for use in beamerthemeconfposter.sty

%-----------------------------------------------------------
% Define the column widths and overall poster size
% To set effective sepwid, onecolwid and twocolwid values, first choose how many columns you want and how much separation you want between columns
% In this template, the separation width chosen is 0.024 of the paper width and a 4-column layout
% onecolwid should therefore be (1-(# of columns+1)*sepwid)/# of columns e.g. (1-(4+1)*0.024)/4 = 0.22
% onecolwid should therefore be (1-(# of columns+1)*sepwid)/# of columns e.g. 
% (1-(3+1)*0.025)/3 = 0.3
% Set twocolwid to be (2*onecolwid)+sepwid = 0.464
% Set threecolwid to be (3*onecolwid)+2*sepwid = 0.708

\newlength{\sepwid}
\newlength{\onecolwid}
\newlength{\twocolwid}
\newlength{\threecolwid}
\setlength{\paperwidth}{36in} % A0 width: 46.8in
\setlength{\paperheight}{48in} % A0 height: 33.1in
\setlength{\textwidth}{34in} % A0 width: 46.8in
\setlength{\textheight}{46in} % A0 height: 33.1in
\setlength{\sepwid}{0.025\paperwidth} % Separation width (white space) between columns
\setlength{\onecolwid}{0.3\paperwidth} % Width of one column
\setlength{\twocolwid}{0.625\paperwidth} % Width of two columns
\setlength{\threecolwid}{0.95\paperwidth} % Width of three columns
\setlength{\topmargin}{-0.5in} % Reduce the top margin size
%-----------------------------------------------------------

\usepackage{graphicx}  % Required for including images
\newcommand{\Cyclus}{\textsc{Cyclus}\xspace}%

\usepackage{tabularx}
\newcolumntype{b}{X}
\newcolumntype{s}{>{\hsize=.5\hsize}X}
\newcolumntype{m}{>{\hsize=.75\hsize}X}
\newcolumntype{z}{>{\hsize=.65\hsize}X}

\usepackage{booktabs} % Top and bottom rules for tables
\usepackage{xspace}


\setbeamertemplate{bibliography item}[text]

%----------------------------------------------------------------------------------------
%	TITLE SECTION 
%----------------------------------------------------------------------------------------

\title{
	\includegraphics[width=0.3\linewidth]{UIUC_Logo}
	\vspace{2cm}
	\hspace{50cm}
	\vspace{1cm}
	Simulation of Spent Nuclear Fuel loading into a Final Waste Repository
} % Poster title

\author{\textbf{Gwendolyn J. Chee}, Kathryn D. Huff}
\institute{University of Illinios at Urbana-Champaign, Department of Nuclear, Plasma, and Radiological Engineering, Urbana, IL 61801}
%----------------------------------------------------------------------------------------

\begin{document}

\addtobeamertemplate{block end}{}{\vspace*{2ex}} % White space under blocks
\addtobeamertemplate{block alerted end}{}{\vspace*{2ex}} % White space under highlighted (alert) blocks

\setlength{\belowcaptionskip}{2ex} % White space under figures
\setlength\belowdisplayshortskip{2ex} % White space under equations

\begin{frame}[t] % The whole poster is enclosed in one beamer frame

\begin{columns}[t,totalwidth=\threecolwid] % The whole poster consists of three major columns, the second of which is split into two columns twice - the [t] option aligns each column's content to the top

\begin{column}{0.5\sepwid}\end{column} % Empty spacer column

\begin{column}{\onecolwid} % The first column

%----------------------------------------------------------------------------------------
%	OBJECTIVES
%----------------------------------------------------------------------------------------
\setbeamercolor{block alerted title}{fg=black,bg=norange} % Change the alert block title colors
\setbeamercolor{block alerted body}{fg=black,bg=white} % Change the alert block body colors
\begin{alertblock}{Objectives}
\begin{itemize}
        \item Create a \Cyclus Waste Conditioning Archetype that packages spent fuel 
        bundles into canisters which have user-defined properties. 
		\item Create a \Cyclus Medium-Fidelity Repository Archetype that gives accurate 
		time and spatial dependent temperature calculations and loads the repository
		based on a user-selected loading strategy. 
\end{itemize}

\end{alertblock}

%----------------------------------------------------------------------------------------
%	MOTIVATION
%----------------------------------------------------------------------------------------

\begin{block}{Motivation}
Previous work in studying repository loading have used spent fuel assemblies 
that have an average burn up composition \cite{johnson_optimizing_2016} 
to evaluate the heat load in the repository \cite{greenberg_application_2012}.
Instead of using average \gls{SNF} composition, this work aims to use \textbf{U.S. 
historical SNF inventory data} \cite{peterson_unf_standards_2017} in various 
simulations to study how waste acceptance strategies impact repository loading.

\end{block}

%----------------------------------------------------------------------------------------
%	Cyclus
%----------------------------------------------------------------------------------------

\begin{block}{Cyclus}
In \textsc{Cyclus}, each facility in the fuel cycle is modeled individually 
and they interact with one another as independent \textit{agents}. 
The goal of this work is to develop a waste conditioning \textit{agent} 
and a waste repository \textit{agent} to provide \textsc{Cyclus} with 
the capabilities to run these simulations. 
\begin{figure}
	\includegraphics[width=0.9\linewidth]{Cyclus_graph}
	\caption{Cyclus API allows for modular build of simulations \cite{huff_2018}}
\end{figure}

\end{block}
%----------------------------------------------------------------------------------------

\end{column} % End of the first column

\begin{column}{\sepwid}\end{column} % Empty spacer column


%----------------------------------------------------------------------------------------

\begin{column}{\onecolwid} % The second column
%----------------------------------------------------------------------------------------
%	MODELS
%----------------------------------------------------------------------------------------

\begin{block}{Waste Conditioning Model}
The waste conditioning model accepts spent fuel bundles and puts them into a cylindrical waste canister. 
User-defined Variables: 
\begin{itemize}
	\item Number of spent fuel bundles in each canister 
	\item Thermal Conductivity of Canister 
	\item Thermal diffusivity of Canister 
	\item Radius and height of Canister
\end{itemize}
\end{block}

\begin{block}{Waste Repository Model}
The waste repository model accepts

    
%----------------------------------------------------------------------------------------
%	PREVIOUS WORK
%----------------------------------------------------------------------------------------

\begin{block}{Archetype Design}


\end{block}

%----------------------------------------------------------------------------------------

\end{column} % End of column 2

\begin{column}{\sepwid}\end{column} % Empty spacer column

\begin{column}{\onecolwid} % The third column
	
%----------------------------------------------------------------------------------------
%	FUTURE WORK
%----------------------------------------------------------------------------------------

\begin{alertblock}{Future Work}


\end{alertblock}


%--------------------------------------------------------
% PROJECT TIMELINE
%---------------------------------------------------------
\begin{block}{Timeline}
\end{block}



%----------------------------------------------------------------------------------------
%	ACKNOWLEDGEMENTS
%----------------------------------------------------------------------------------------

\setbeamercolor{block title}{fg=norange,bg=white} % Change the block title color

\begin{block}{Acknowledgements}
	
	This research was performed using funding received
	from the Consortium for Nonproliferation Enabling
	Capabilities under award number 1-483313-973000-191100.
	
	
\end{block}

%----------------------------------------------------------------------------------------
%	CONTACT INFORMATION
%----------------------------------------------------------------------------------------

\setbeamercolor{block alerted title}{fg=black,bg=norange} % Change the alert block title colors
\setbeamercolor{block alerted body}{fg=black,bg=white} % Change the alert block body colors



\begin{block}{References}

        \footnotesize\bibliographystyle{abbrv} 
        \bibliography{poster}
\end{block}


%----------------------------------------------------------------------------------------



\end{column} % End of the third column

\end{columns} % End of all the columns in the poster

\end{frame} % End of the enclosing frame

\end{document}
\begin{column}{\sepwid}\end{column} % Empty spacer column
