%%%%%%%%%%%%%%%%%%%%%%%%%%%%%%%%%%%%%%%%%
% Jacobs Landscape Poster
% LaTeX Template
% Version 1.1 (14/06/14)
%
% Created by:
% Computational Physics and Biophysics Group, Jacobs University
% https://teamwork.jacobs-university.de:8443/confluence/display/CoPandBiG/LaTeX+Poster
% 
% Further modified by:
% Nathaniel Johnston (nathaniel@njohnston.ca)
%
% This template has been downloaded from:
% http://www.LaTeXTemplates.com
%
% License:
% CC BY-NC-SA 3.0 (http://creativecommons.org/licenses/by-nc-sa/3.0/)
%
%%%%%%%%%%%%%%%%%%%%%%%%%%%%%%%%%%%%%%%%%

%----------------------------------------------------------------------------------------
%	PACKAGES AND OTHER DOCUMENT CONFIGURATIONS
%----------------------------------------------------------------------------------------

\documentclass[final]{beamer}

\usepackage[scale=1.0]{beamerposter} % Use the beamerposter package for laying out the poster
\usepackage[acronym,toc]{glossaries}
\include{acros}
\usetheme{confposter} % Use the confposter theme supplied with this template

\setbeamercolor{block title}{fg=dblue!80,bg=white} % Colors of the block titles
\setbeamercolor{block body}{fg=black,bg=white} % Colors of the body of blocks
\setbeamercolor{block alerted title}{fg=white,bg=dblue!70} % Colors of the highlighted block titles
\setbeamercolor{block alerted body}{fg=black,bg=dblue!10} % Colors of the body of highlighted blocks
% Many more colors are available for use in beamerthemeconfposter.sty

%-----------------------------------------------------------
% Define the column widths and overall poster size
% To set effective sepwid, onecolwid and twocolwid values, first choose how many columns you want and how much separation you want between columns
% In this template, the separation width chosen is 0.024 of the paper width and a 4-column layout
% onecolwid should therefore be (1-(# of columns+1)*sepwid)/# of columns e.g. (1-(4+1)*0.024)/4 = 0.22
% onecolwid should therefore be (1-(# of columns+1)*sepwid)/# of columns e.g. 
% (1-(3+1)*0.025)/3 = 0.3
% Set twocolwid to be (2*onecolwid)+sepwid = 0.464
% Set threecolwid to be (3*onecolwid)+2*sepwid = 0.708

\newlength{\sepwid}
\newlength{\onecolwid}
\newlength{\twocolwid}
\newlength{\threecolwid}
\setlength{\paperwidth}{36in} % A0 width: 46.8in
\setlength{\paperheight}{48in} % A0 height: 33.1in
\setlength{\textwidth}{34in} % A0 width: 46.8in
\setlength{\textheight}{46in} % A0 height: 33.1in
\setlength{\sepwid}{0.025\paperwidth} % Separation width (white space) between columns
\setlength{\onecolwid}{0.3\paperwidth} % Width of one column
\setlength{\twocolwid}{0.625\paperwidth} % Width of two columns
\setlength{\threecolwid}{0.95\paperwidth} % Width of three columns
\setlength{\topmargin}{-0.5in} % Reduce the top margin size
%-----------------------------------------------------------

\usepackage{graphicx}  % Required for including images
\newcommand{\Cyclus}{\textsc{Cyclus}\xspace}%
\usepackage{tabularx}
\newcolumntype{b}{X}
\newcolumntype{s}{>{\hsize=.5\hsize}X}
\newcolumntype{m}{>{\hsize=.75\hsize}X}
\newcolumntype{z}{>{\hsize=.65\hsize}X}

\usepackage{booktabs} % Top and bottom rules for tables
\usepackage{xspace}
\usepackage{amsmath}
\usepackage{exscale}

\setbeamertemplate{bibliography item}[text]

%----------------------------------------------------------------------------------------
%	TITLE SECTION 
%----------------------------------------------------------------------------------------

\title{%
  \texorpdfstring{%
    \makebox[\linewidth]{%
      \makebox[0pt][l]{%
        \raisebox{\dimexpr-\height+\baselineskip}[0pt][0pt]
          {\includegraphics[height=2.5\baselineskip]{UIUC_Logo}}% Left logo
      }\hfill
      \makebox[0pt]{Simulation of Spent Nuclear Fuel loading}%
      \hfill\makebox[0pt][r]{%
        \raisebox{\dimexpr-\height+\baselineskip}[0pt][0pt]
          {\includegraphics[height=3.3\baselineskip]{arfc_atom}}% Right logo
      }%
    }%
  }
  {Simulation of Spent Nuclear Fuel loading}
  {into a Final Waste Repository}
  {\vspace{1cm}}
  } % Poster title

\author{\textbf{Gwendolyn J. Chee}, Kathryn D. Huff}
\institute{University of Illinios at Urbana-Champaign, Department of Nuclear, Plasma, and Radiological Engineering, Urbana, IL 61801}
%----------------------------------------------------------------------------------------

\begin{document}

\addtobeamertemplate{block end}{}{\vspace*{2ex}} % White space under blocks
\addtobeamertemplate{block alerted end}{}{\vspace*{2ex}} % White space under highlighted (alert) blocks

\setlength{\belowcaptionskip}{2ex} % White space under figures
\setlength\belowdisplayshortskip{2ex} % White space under equations

\begin{frame}[t] % The whole poster is enclosed in one beamer frame

\begin{columns}[t,totalwidth=\threecolwid] % The whole poster consists of three major columns, the second of which is split into two columns twice - the [t] option aligns each column's content to the top

\begin{column}{0.5\sepwid}\end{column} % Empty spacer column

\begin{column}{\onecolwid} % The first column

%----------------------------------------------------------------------------------------
%	MOTIVATION
%----------------------------------------------------------------------------------------

\begin{block}{Introduction}

\textbf{Previous Work}

Previous work towards studying repository loading have used: 
\begin{itemize}
	\item \gls{SNF} with an \textbf{average burnup composition} 
	    \cite{greenberg_application_2012,johnson_optimizing_2016}
	\item a \textbf{lumped capacitance thermal model} for calculating temperature 
	      in a \Cyclus repository model \cite{huff_rapid_2017}
\end{itemize}

\vspace{0.7em}
\textbf{Motivation}

The goal of this work is to improve on the repository models 
and use U.S. historical SNF inventory data in simulations 
to more accurately study the loading of a waste repository. 

These goals will be achieved by: 
\begin{itemize}
\item using UNF-ST$\&$DARDS \gls{UDB} \cite{peterson_unf_standards_2017} that has 
\textbf{historic assembly-specific data} (e.g, isotopic composition, heat) in \Cyclus simulations 
\item implementing a \textbf{more accurate thermal model} within a \Cyclus repository model 
\end{itemize}

\end{block}

%----------------------------------------------------------------------------------------
%	OBJECTIVES
%----------------------------------------------------------------------------------------
\setbeamercolor{block alerted title}{fg=black,bg=norange} % Change the alert block title colors
\setbeamercolor{block alerted body}{fg=black,bg=white} % Change the alert block body colors
\begin{alertblock}{Objectives}
\begin{itemize}
        \item Create a \Cyclus spent fuel conditioning model that packages spent fuel 
        bundles into packages which have user-defined properties. 
		\item Create a \Cyclus medium-fidelity repository model that gives accurate 
		time and spatial dependent temperature values and loads the repository
		based on a user-selected loading strategy. 
\end{itemize}

\end{alertblock}

%----------------------------------------------------------------------------------------
%	Cyclus
%----------------------------------------------------------------------------------------

\begin{block}{Cyclus}
\Cyclus is an agent-based extensible framework for modeling flow of material 
through user-defined nuclear fuel cycles \cite{huff_fundamental_2016}. 
In \Cyclus, each facility in the fuel cycle is modeled individually 
and the facilities interact with one another as independent \textit{agents}. 
\begin{figure}
	\includegraphics[width=0.9\linewidth]{Cyclus_graph}
	\caption{\Cyclus API allows for modular build of simulations \cite{huff_fundamental_2016}}
\end{figure}

\end{block}

\begin{block}{Spent Fuel Conditioning Model}
The spent fuel conditioning model accepts spent fuel bundles and puts them into a cylindrical
waste package. 
	
In the spent fuel conditioning model, the user can define variables:  

For each layer, 
\begin{itemize}
	\item radius 
	\item thermal conductivity 
	\item thermal diffusivity
\end{itemize}
For each package,
\begin{itemize}
	\item Number of spent fuel bundles
	\item Radius and height
\end{itemize}

\end{block}

%----------------------------------------------------------------------------------------

\end{column} % End of the first column

\begin{column}{\sepwid}\end{column} % Empty spacer column


%----------------------------------------------------------------------------------------

\begin{column}{\onecolwid} % The second column
%----------------------------------------------------------------------------------------
%	MODELS
%----------------------------------------------------------------------------------------

\begin{block}{Waste Repository Model}
The waste repository model accepts waste packages and emplaces them into 
available positions within the waste repository based on a thermal criteria.  
The thermal criteria is a temperature limit at the interface between the waste 
package surface and the host geology, that is set based on the repository's host geology.

\begin{table}[]
	\label{tab:temp_limit}
	\caption{Temperature limit at waste package surface, thermal conductivity 
	and thermal diffusivity for each host geology \cite{sutton_investigations_2011}}
	\begin{tabular}{|l|l|l|l|}
	\hline
	Rock Type & $T_{limit}$ [$^\circ$C] & k [$\frac{W}{mK}$] &  $\alpha$ [$\frac{m^2}{s}$]  \\ \hline
	Granite   & 100 & 2.5  & 1.13\\ \hline
	Clay      & 100 & 1.75 & 6.45\\ \hline
	Salt      & 200 & 4.2  & 2.07\\ \hline
	\end{tabular}
\end{table}

In the waste repository model, the user can define the variables: 
	\begin{itemize}
		\item Capacity
		\item Distance between waste packages
		\item Distance between drifts 
		\item Repository host geology 
		\item Loading Strategy 
	\end{itemize}

\vspace{0.7em}
\textbf{Thermal Model}

After the addition of new waste packages at each time step, the waste repository model 
recalculates the temperature at each location in the repository. 
If the addition of this new package causes its temperature to exceed the thermal 
limit, it will be placed back into the buffer. 
A thermal model that relies on a transient `outside' model and quasi-steady-state 
`inside' model is used to accurately determine the temperature in the repository
\cite{sutton_investigations_2011}. 

\vspace{0.7em}
\textbf{Transient `Outside' Model}

The `outside' model assumes a homogenous medium with the \gls{EBS} replaced by the 
geologic medium. 
Figure \ref{fig:conceptual_layout} shows the conceptual layout of the central waste 
package and the adjacent point and line sources. 

\begin{figure}
	\label{fig:conceptual_layout}
	\includegraphics[width=1\linewidth]{outsidemodel}
	\caption{`Outside' Model: Conceptual layout of the central waste package, its adjacent
	point sources and adjacent line sources \cite{sutton_investigations_2011}}
\end{figure}

Temperature solutions for the central waste package, adjacent point and line sources 
are superimposed to calculate the temperature at specific points in the repository.
The equations for calculating temperature of each contributing component 
are included below \cite{sutton_investigations_2011,greenberg_application_2012,huff_numerical_2012}. 

The central drift consists of one finite line source which represents the central 
waste package.
\begin{align*}
	T_{line}(t,x,y,z) &= \frac{1}{8 \pi k}  \int _{0}^{t} \frac{q_L(t')}{t-t'}e^{\frac{-(x^2+z^2)}{4\alpha(t-t')}} \\
	&[erf[\frac{1}{2}\frac{y+L/2}{\sqrt{\alpha(t-t')}}]-erf[\frac{1}{2}\frac{y-L/2}{\sqrt{\alpha(t-t')}}]] dt'
\end{align*}
The central drift also consists of point sources that represent neighboring 
waste packages in the central drift. 
\begin{align*}
	T_{point}(t,r) = \frac{1}{8 k \sqrt{\alpha} \pi^{3/2}} \int_{0}^{t}\frac{q(t')}{(t-t')^{3/2}}e^{\frac{-r^2}{4\alpha(t-t')}}dt'
\end{align*}
The neighboring drifts are represented by infinite line sources.  
\begin{align*}
	T_{\infty line}(t,x,z) = \frac{1}{4\pi k} \int_0^t \frac{q_L(t')}{t-t'} e^{\frac{-(x^2+z^2)}{4\alpha (t-t')}} dt'
\end{align*}

\end{block}


%----------------------------------------------------------------------------------------

\end{column} % End of column 2

\begin{column}{\sepwid}\end{column} % Empty spacer column

\begin{column}{\onecolwid} % The third column

\begin{block}{Waste Repository Model}
\textbf{Quasi-Steady-State `Inside' Model}

The `inside' model is considered to be at a quasi-steady-state condition because \gls{EBS}
has a relatively low thermal mass compared to the infinite geologic medium 
\cite{sutton_investigations_2011}. 
The steady state calculation is performed at each time step with the heat source and 
interface temperature as boundary conditions. 
Figure \ref{fig:ebs_layers} illustrates an \gls{EBS} layout.

\begin{figure}
	\label{fig:ebs_layers}
	\includegraphics[width=0.9\linewidth]{ebs_layers}
	\caption{Layers in an Engineering Barrier System \cite{sutton_investigations_2011}}
\end{figure}

\end{block}


\setbeamercolor{block alerted title}{fg=black,bg=norange} % Change the alert block title colors
\setbeamercolor{block alerted body}{fg=black,bg=white} % Change the alert block body colors
\begin{alertblock}{Future Work }
\begin{itemize}
		\item Run \Cyclus simulations with U.S. historical SNF inventory data, the spent fuel conditioning 
		and repository models to study how waste package acceptance strategies impact repository loading
\end{itemize}

\end{alertblock}


%----------------------------------------------------------------------------------------
%	ACKNOWLEDGEMENTS
%----------------------------------------------------------------------------------------

\setbeamercolor{block title}{fg=norange,bg=white} % Change the block title color

\begin{block}{Acknowledgements}
	
	This research is being performed using funding received from 
	the DOE Office of Nuclear Energy's Nuclear Energy University 
	Program (Project 16-10512, DE-NE0008567) 'Demand-Driven Cycamore 
	Archetypes'.
	
	
\end{block}

%----------------------------------------------------------------------------------------
%	CONTACT INFORMATION
%----------------------------------------------------------------------------------------

\setbeamercolor{block alerted title}{fg=white,bg=dblue} % Change the alert block title colors
\setbeamercolor{block alerted body}{fg=black,bg=white} % Change the alert block body colors

\begin{alertblock}{Contact Information}
	\setbeamercolor{block title}{fg=norange,bg=white} % Change the block title color
	\begin{itemize}
		\item Email: \href{mailto:gchee2@illinois.edu}{gchee2@illinois.edu}
	\end{itemize}
	
\end{alertblock}

\begin{block}{References}

	{\footnotesize\bibliographystyle{abbrv} 
	\bibliography{poster}}
\end{block}


%----------------------------------------------------------------------------------------



\end{column} % End of the third column

\end{columns} % End of all the columns in the poster

\end{frame} % End of the enclosing frame

\end{document}
\begin{column}{\sepwid}\end{column} % Empty spacer column
